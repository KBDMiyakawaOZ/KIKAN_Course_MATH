%% 定理,系の番号付けサンプル
\documentclass[dvipdfmx,VUP,useotf,usehyperref]{CoreCourseMath}
\usepackage[scale=1,deluxe,jis2004]{otf}
\usepackage[MS]{kbdaddjfont}\let\fmb\mcb\let\textfmb\textmcb
\usepackage{TkizasciiMP}
\usepackage{ccMath}
\begin{document}
\setcounter{chapter}{1}
\begin{命題}
○○○○○○○○○○○○○○○○○○○○○○○○○○○○○○○○○○○○○○○○○○
○○○○○○○○○○○○○○○○○○○○○○○○○○○○○○○○○○○○○○○○○○
\end{命題}
「定理」の入力は\verb|\begin{定理}{LABEL名}[副題]|となります\footnote{\textsf{[副題]}はオプションです.(以下同)}.
\begin{定理}{LABEL名}[副題]
○○○○○○○○○○○○○○○○○○○○○○○○○○○○○○○○○○○○○○○○○○
○○○○○○○○○○○○○○○○○○○○○○○○○○○○○○○○○○○○○○○○○○
\end{定理}
「系」の入力は\verb|\begin{系}{LABEL名}[副題]|となり,\textsf{LABEL名}は,親の「定理」と同じものを指定します.
%% 「系」{LABEL}
\begin{系}{LABEL名}[副題]
○○○○○○○○○○○○○○○○○○○○○○○○○○○○○○○○○○○○○○○○○○
○○○○○○○○○○○○○○○○○○○○○○○○○○○○○○○○○○○○○○○○○○
\end{系}
\begin{定理}{LABEL名B}[副題]
○○○○○○○○○○○○○○○○○○○○○○○○○○○○○○○○○○○○○○○○○○
○○○○○○○○○○○○○○○○○○○○○○○○○○○○○○○○○○○○○○○○○○
\end{定理}

\begin{補題}
○○○○○○○○○○○○○○○○○○○○○○○○○○○○○○○○○○○○○○○○○○
\end{補題}
孫の「系」は,\verb|\begin{系*}{a}{LABEL名}[副題]|のように\verb+*+を付加します.
このうち\verb|#1|は印字されるアルファベットの枝番,\verb|#2|の\textsf{LABEL名}は,子の「系」と同様に親の「定理」と同じ\textsf{LABEL名}を入力します.
%% 枝番「系」
\begin{系*}{a}{LABEL名}[副題]
○○○○○○○○○○○○○○○○○○○○○○○○○○○○○○○○○○○○○○○○○○
○○○○○○○○○○○○○○○○○○○○○○○○○○○○○○○○○○○○○○○○○○
\end{系*}
\verb|\begin{系*}{b}{LABEL名}[副題]|
\begin{系*}{b}{LABEL名}[副題]
○○○○○○○○○○○○○○○○○○○○○○○○○○○○○○○○○○○○○○○○○○
○○○○○○○○○○○○○○○○○○○○○○○○○○○○○○○○○○○○○○○○○○
\end{系*}
本文参照は,定理の引用``定理\ref{LABEL名}``,
子の系の引用``系\ref{LABEL名}``,孫は``系\ref{LABEL名}a`` というイメージです.
\chapter{}

\begin{命題}
○○○○○○○○○○○○○○○○○○○○○○○○○○○○○○○○○○○○○○○○○○
○○○○○○○○○○○○○○○○○○○○○○○○○○○○○○○○○○○○○○○○○○
\end{命題}
\begin{定理}{Ch2:LABEL名}[副題]
○○○○○○○○○○○○○○○○○○○○○○○○○○○○○○○○○○○○○○○○○○
○○○○○○○○○○○○○○○○○○○○○○○○○○○○○○○○○○○○○○○○○○
\end{定理}
としておき,子の系は

\begin{系}{Ch2:LABEL名}[副題]
○○○○○○○○○○○○○○○○○○○○○○○○○○○○○○○○○○○○○○○○○○
\end{系}
孫の系は

\begin{定理}{LABEL名B}[副題]
○○○○○○○○○○○○○○○○○○○○○○○○○○○○○○○○○○○○○○○○○○
○○○○○○○○○○○○○○○○○○○○○○○○○○○○○○○○○○○○○○○○○○
\end{定理}

\begin{補題}
○○○○○○○○○○○○○○○○○○○○○○○○○○○○○○○○○○○○○○○○○○
○○○○○○○○○○○○○○○○○○○○○○○○○○○○○○○○○○○○○○○○○○
\end{補題}

\begin{系*}{a}{Ch2:LABEL名}[副題]
○○○○○○○○○○○○○○○○○○○○○○○○○○○○○○○○○○○○○○○○○○
○○○○○○○○○○○○○○○○○○○○○○○○○○○○○○○○○○○○○○○○○○
\end{系*}
\begin{系*}{b}{Ch2:LABEL名}[副題]
○○○○○○○○○○○○○○○○○○○○○○○○○○○○○○○○○○○○○○○○○○
\end{系*}
本文参照は,定理の引用``定理\ref{Ch2:LABEL名}``,
子の系の引用``系\ref{Ch2:LABEL名}``,孫は``系\ref{Ch2:LABEL名}a`` というイメージです.

\end{document}

